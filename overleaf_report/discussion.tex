\section{Discussion}
\label{sec:discussion}
This section will motivate the expansion of research into the energy impact of running the robotics software itself.
As described numerously in this literature study, this study aimed to give insights into the state-of-the-art of research
on the energy efficiency impact of software aspects in robotics software.
However, it became quickly apparant after the initial search that such studies were not numerous.
Out of the 683 potentially relevant studies, 17 primary studies were selected based on the application of the selection criteria, 
from which two studies would classify as a study into the energy efficiency impact of the robotics software itself.

Their publication dates are 2017 \cite{hou2017novel_cloud_evaluation_model} and 2019 \cite{rahman2019cloud_robot_offloading}, 
proving the infancy of the field of study.

The importance of such research is significant, considering the high emmisions and use of energy in the automation sector, as explained
in section \ref{sec:intro}.
This literature study has mainly studied papers improving the energy efficiency of physical inefficiencies, 
be it loss of traction (physical), the use of an FPGA\footnote{Field Programmable Gate Array} in combination with a neural network
to improve the acceleration on-board (hardware) or the improvement of an inefficient path finder (software).

One could thus say with confidence that, while there is still much to gain from that field, the impact of software itself is mostly left out
of the picture. 
As our society puts a growing effort towards a sustainable future, it warrants the expansion of research into this field of study.
The biggest improvement in energy efficiency is going to be achieved by combining all the aforementioned state-of-the-art methods for improving 
energy efficiency in section \ref{sec:results:rq2_state_of_the_art}, with software that is designed to be as energy efficient as possible.
This is already partly possible because of the contributions of \cite{hou2017novel_cloud_evaluation_model}, however the fact that this is the only paper
out of 17 primary studies, selected from 683 potentially relevant studies, which explicitly focuses on the identification of energy inefficient software, should warrant motivation for
the expansion of such research.