\section{Discussion}
\label{sec:discussion}
This literature study initially set out to research the state-of-the-art in terms of improving the
energy efficiency of robotics software \textit{itself}.
The fact that only two of the primary studies presented such research forms the basis of this discussion.
Their publication dates are 2017 \cite{hou2017novel_cloud_evaluation_model} and 2019 \cite{rahman2019cloud_robot_offloading}, 
indicating the infancy of the field of study. The importance of such research is significant, 
considering the high emmisions and use of energy in the automation sector, as explained in section \ref{sec:intro}.
This literature study has mainly studied papers improving the energy efficiency of existing inefficiencies, 
be it the loss of traction (physical) \cite{kim2016firefighting_robot}, 
the use of an FPGA in combination with a neural network
to improve the acceleration on-board (hardware) \cite{cheng2018FPGA_image_recognition} 
or the improvement of an inefficient path finder (software) \cite{mei2006mobile_exploration}.

One could thus say with confidence that, while there is still much to gain from that field of study,
the impact of robotics software itself on energy efficiency is not yet sufficiently researched.
As our society puts a growing effort towards a sustainable future, it warrants the expansion of research into this field of study.
One could reason that an improvement in energy efficiency can be achieved by combining all the aforementioned state-of-the-art methods for improving 
energy efficiency in section \ref{sec:results:rq2_state_of_the_art}, with software that is designed to be as energy efficient as possible.
However, proof of such a claim is needed.
A good start towards creating energy efficient software has already been provided; 
a novel green evaluation method for cloud robotics \cite{hou2017novel_cloud_evaluation_model}.
Which enables the identification of bottlenecks in robotics software in terms of energy efficiency, 
and the prediction of the energy efficiency of a piece of software during development.
However, the fact that this is the only study, out of 17 primary studies, selected from 683 potentially relevant studies, 
which explicitly focuses on the identification of energy inefficient software, should warrant motivation for
the expansion of such research.