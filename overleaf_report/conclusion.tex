\section{Conclusions}\label{sec:conclusions}
To conclude this literature study; the waste of energy by persisting inefficiencies in robotic hardware, software or the physical world
(e.g. the weight of payload) can be significantly improved by applying the findings of this literature study as set out in section 
\ref{sec:results}, specifically subsection \ref{sec:results:rq2_state_of_the_art}.
Despite the improvements in CO2 emmissions and energy consumption over the past years, as mentioned in section \ref{sec:intro}, 
Fysikopoulos et al. \cite{fysikopoulos2012automotive_energy_consumption} assert that 20\% to 40\% unnecessary use of energy may 
still be found in industrial firms.

\vspace{5mm}

The unnecessary use of energy can only be significantly reduced by combining the tactics for solving known inefficiencies in the field of robotics,
as this literature study has studied and presented as its main findings, with the future research into the impact on energy efficiency by the
robotics software itself.
The motivation for the expansion into this field of study has been given in section \ref{sec:discussion} and is the most important
contribution of this literature study.