\section{Threats To Validity}\label{sec:threats}
Report about each type of threat to the validity of the experiment, according to the classification discussed in class.

\subsection{Internal Validity}
This threat has been mitigated as much as possible by defining the research protocol, as explained in section \ref{sec:study_design}, 
as rigorously as possible.
Itertivly defining it by discussing it after each iteration with my supervisor\footnote{dr. Ivano Malavolta, Vrije Universiteit Amsterdam}

\subsection{External Validity}
The most severe potential external threat to the validity of this literature study is that the primary studies would not 
be representative of the state-of-the-art on energy efficiency in robotics software.

To avoid this to happen, the search strategy applied consisted of both automatic search and backward and forward snowballing 
\cite{wohlin2014snowballing}. Specifically, the presence of potential gaps left out by the automatic search was mitigated by 
the snowballing technique. 

This enlarged the set of relevant studies by considering each study selected in the automatic search, 
and focussing on those papers either citing or cited by it. 
Also, only peer-reviewed papers were considered and secondary and tertiary studies exlcuded.
This potential bias did not significantly impact this literature study, since the considered papers have undergone a rigorous
peer-review process, which is a well-established requirement for high quality publications. 

The inclusion and exclusion criteria were also rigorously and iteratively defined, again discussing it with my supervisor,
before the study design was put into action.

\subsection{Construct Validity}
This potential bias was mitigated by automatically searching the studies on the any data source as indexed by \textbf{Google Scholar}.
The search string was also kept as general as possible, as explained before in section \ref{sec:study_design:search_selection}.

\subsection{Conclusion Validity}
Potential biases during the data extraction process were mitigated by rigorously and iteratively defining the data sheet with my superivsor.
By doing so, the alignment of the data extraction process with the research questions was guaranteed.
Furthermore, any potential threats to conclusion validity were mitigated, in general, by applying the best practices on systematic
literature reviews in each phase of this study, as stated in 
\cite{petersen2015guidelines_systematic, kitchenham2013systematic_review_guidelines, wohlin2012experimentation}.

This makes this literature study easy to be checked and replicated by other researchers.