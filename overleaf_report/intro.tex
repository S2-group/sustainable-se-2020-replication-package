\section{Introduction}
\label{sec:intro}

% quote from paper 16
% Explains mobile robots and their complications with energy consumption
Mobile robots are widely used in many applications \cite{mei2005energy_consumers_identified}.
People can buy intelligent robotic vacuum cleaners or lawn mowers from stores. 
Some hospitals are using robots to provide quick and safe medicine delivery \cite{evans1994courier_hospital}.

Batteries are often used to provide power for mobile robots; however, they are heavy to carry and have limited energy capacity. 
A Honda humanoid robot can walk for only 30 minutes with a battery pack they carry on the back \cite{aylett2002machines_to_life}; energy is the most important challenge for mobile robots. 
Rybski et al. \cite{rybsky2000robot_rangers} show that power consumption is one of the major issues in their robot design.

% Explain Industry4.0 and the impact on energy consumption and society
Robots can also be non-mobile, these robots mostly exist in an industrial setting and form the basis of the fourth industrial revolution,
also called \textit{Industry 4.0} \cite{lasi2014industry4}. 
% Prime examples of multi-robot, Industry 4.0 applications include: material handling \cite{wan2017iot_material_handling}, assembly line \cite{stenmark2015assemblyline_robots}, warehouse maintenance, cooperative navigation \cite{muhlbacher2017multi_robot_navigation} etc.
Industrial firms contribute to 36\% of total global energy consumption and 24\% of total CO2 emissions \cite{international2006energy}.
% Nowadays, the society is putting a growing effort towards reducing greenhouse gas emissions and to reach an environmentally 
% sustainable future. This awareness has brought energy consumption and CO2 emission related objectives to the attention of operations managers in manufacturing companies. 
% In production scheduling decisions, besides time related objectives, energy consumption related objectives have become important \cite{gurel2019industrial_robot_scheduling}. 
Energy consumption in the manufacturing sector has been declining since 1998. 
For instance, in the U.S., the energy consumption in the manufacturing sector decreased by 17\% from 2002 to 2010 \cite{US2018energy_administration}.
Despite these improvements, Fysikopoulos et al. \cite{fysikopoulos2012automotive_energy_consumption} assert that 20\% to 40\% unnecessary use of energy may still be found in industrial firms. 
Hence the energy performance of manufacturing systems is a \textit{major area of research} and a concern for many manufacturing companies.

According to the IFR Statistical Department \cite{IFR2010executive_summary}, the level of automation in the automobile frame- and body construction process was 90\%, which implies a heavy use of industrial robots in related tasks. 
Also, Engelmann \cite{engelmann2009energy_efficient_factories} states that about 8\% of the total energy consumption in automotive industries belongs to industrial robots. \\

% Explain motivation and goal for this paper and what is explained in what section
Considering the aforementioned, it is logical to understand that the effort to maximize energy efficiency in robotics will have a significant impact on the world energy consumption and consequently CO2 emissions.

The \textbf{goal} of this study is to present an approachable survey of existing research on energy efficiency in robotics software in order to motivate expansions to research the energy impact of the robotics software itself.
Considering robotics, it is to be expected that the distinction between the physical and non-physical, the software and hardware, is somewhat blurred; \textit{e.g. improving the loss of traction (and thus the loss of energy efficiency) of a robot by its payload weight, by adding sub-robots carrying the payload. 
Which can only work together with an elaborate, novel, distributed systems algorithm}.
This blurred distinction complicated this literature study, as will be apparent in section \ref{sec:results}, and forms the basis for the motivation for expansion of research into the energy impact of robotics software itself - as explained in section \ref{sec:discussion}.

For this study a total set of 683 potentially relevant studies were identified. After the application of the study design, as described in section \ref{sec:study_design}, the set of \textit{primary studies} consisted of 17 studies.