\documentclass[10pt, conference, compsocconf]{IEEEtran}

\usepackage[bookmarks=true]{hyperref}
\usepackage{epsfig}
\usepackage{amsmath,amssymb,amsfonts,latexsym}
\usepackage{enumerate}
\usepackage{xspace}
\usepackage{epsf,picinpar}
\usepackage{varioref}
\usepackage{colortbl,multirow,hhline}
\usepackage{listings}
\usepackage{amssymb}
\usepackage{colortbl,multirow,hhline}
\usepackage{algorithmic}
\usepackage{algorithm}
\usepackage{caption}
\usepackage[normalem]{ulem}
\usepackage{xcolor}
\usepackage{pifont}
\usepackage{xcolor,colortbl}
\usepackage{url}
\usepackage{balance}
\usepackage{graphicx, subfigure}
\usepackage{longtable}
\usepackage{lscape}
\usepackage{multirow}
\usepackage{listings}
\usepackage{framed}
\usepackage{morefloats}
\usepackage[T1]{fontenc}
\usepackage{array}
\usepackage{pdfpages}
\usepackage{fancybox}
\usepackage{amsmath}
\usepackage{flushend}
\usepackage{booktabs}
\usepackage{enumitem}
\usepackage{adjustbox}
\usepackage[charter]{mathdesign}
\usepackage{eulervm}

\renewcommand{\ttdefault}{cmr}

\newcommand{\limit}[1]{\textcolor{red}{\ding{46}~Page limit:~#1}\\}
\newcommand{\todo}[1]{\textcolor{blue}{\ding{46}~#1}} 
\newcommand{\ie}{\emph{i.e.,}\xspace}
\newcommand{\eg}{\emph{e.g.,}\xspace}
\newcommand{\etc}{etc.\xspace}
\newcommand{\etal}{\emph{et~al.}\xspace} 
    
\begin{document}

\title{
	{A Systematic Literature Review on Energy Efficiency in Robotics Software}
}

\author{
\IEEEauthorblockN{dr. Ivano Malavolta}
\IEEEauthorblockA{	
	Supervisor \\ 
	Vrije Universiteit Amsterdam
}

\vspace{5mm}

\IEEEauthorblockN{Stan Swanborn}
\IEEEauthorblockA{S2627602\\
s.o.swanborn@student.vu.nl}
}


\maketitle

\begin{abstract}
\noindent \textit{Context}. 
Nowadays, mobile robots are widely used in many applications. Non-mobile robots are widely used in industrial automation.
This literature study covers the field of research into the energy efficiency of robotics software.

\noindent \textit{Goal}. 
The goal of this literature study is to present a survey of existing research on energy efficiency in robotics software.

\noindent \textit{Method}. 
The method used is a rigorously designed literature study, spanning 17 primary studies. 
These primary studies are summarized, categorized, and analyzed in order to derive relevant patterns across the different studies.

\noindent \textit{Results}. 
The result is given in three parts, as answers to the three research questions. 
The results show that the field of study, while existing since before the change of the century, is still immature.
With attained interest over the last 5 to 10 years.
The results also gave rise to the motivation for the expansion of research into the energy impact of the robotics software itself.
This field of study is still very much in its infancy but proves essential in maximizing energy efficiency of robotic systems, 
as software is the main enabler of robotics, its energy consumption will directly influence the energy consumption level of the
entire system.

\noindent \textit{Conclusions}.
After reading this literature study, the reader will have gained insight in the state-of-the-art of analyzing and 
improving energy efficiency in robotics software. The reader will be able to judge the findings because of the analyzed publication trends
and will be able to judge if the findings are of use for their personal application using the explicate trade-off analysis.
\end{abstract}

\begin{IEEEkeywords}
Green Software, Robotics Software, Mobile Robots, Energy Efficiency, Systematic Literature Review, Systematic Literature Study
\end{IEEEkeywords}

\section{Introduction}
\label{sec:intro}

% quote from paper 16
% Explains mobile robots and their complications with energy consumption
Mobile robots are widely used in many applications \cite{mei2005energy_consumers_identified}.
People can buy intelligent robotic vacuum cleaners or lawn mowers from stores. 
Some hospitals are using robots to provide quick and safe medicine delivery \cite{evans1994courier_hospital}.

Batteries are often used to provide power for mobile robots; however, they are heavy to carry and have limited energy capacity. 
A Honda humanoid robot can walk for only 30 minutes with a battery pack they carry on the back \cite{aylett2002machines_to_life}; energy is the most important challenge for mobile robots. 
Rybski et al. \cite{rybsky2000robot_rangers} show that power consumption is one of the major issues in their robot design.

% Explain Industry4.0 and the impact on energy consumption and society
Robots can also be non-mobile, these robots mostly exist in an industrial setting and form the basis of the fourth industrial revolution,
also called \textit{Industry 4.0} \cite{lasi2014industry4}. 
% Prime examples of multi-robot, Industry 4.0 applications include: material handling \cite{wan2017iot_material_handling}, assembly line \cite{stenmark2015assemblyline_robots}, warehouse maintenance, cooperative navigation \cite{muhlbacher2017multi_robot_navigation} etc.
Industrial firms contribute to 36\% of total global energy consumption and 24\% of total CO2 emissions \cite{international2006energy}.
% Nowadays, the society is putting a growing effort towards reducing greenhouse gas emissions and to reach an environmentally 
% sustainable future. This awareness has brought energy consumption and CO2 emission related objectives to the attention of operations managers in manufacturing companies. 
% In production scheduling decisions, besides time related objectives, energy consumption related objectives have become important \cite{gurel2019industrial_robot_scheduling}. 
Energy consumption in the manufacturing sector has been declining since 1998. 
For instance, in the U.S., the energy consumption in the manufacturing sector decreased by 17\% from 2002 to 2010 \cite{US2018energy_administration}.
Despite these improvements, Fysikopoulos et al. \cite{fysikopoulos2012automotive_energy_consumption} assert that 20\% to 40\% unnecessary use of energy may still be found in industrial firms. 
Hence the energy performance of manufacturing systems is a \textit{major area of research} and a concern for many manufacturing companies.

According to the IFR Statistical Department \cite{IFR2010executive_summary}, the level of automation in the automobile frame- and body construction process was 90\%, which implies a heavy use of industrial robots in related tasks. 
Also, Engelmann \cite{engelmann2009energy_efficient_factories} states that about 8\% of the total energy consumption in automotive industries belongs to industrial robots. \\

% Explain motivation and goal for this paper and what is explained in what section
Considering the aforementioned, it is logical to understand that the effort to maximize energy efficiency in robotics will have a significant impact on the world energy consumption and consequently CO2 emissions.

The \textbf{goal} of this study is to present an approachable survey of existing research on energy efficiency in robotics software in order to motivate expansions to research the energy impact of the robotics software itself.
Considering robotics, it is to be expected that the distinction between the physical and non-physical, the software and hardware, is somewhat blurred; \textit{e.g. improving the loss of traction (and thus the loss of energy efficiency) of a robot by its payload weight, by adding sub-robots carrying the payload. 
Which can only work together with an elaborate, novel, distributed systems algorithm}.
This blurred distinction complicated this literature study, as will be apparent in section \ref{sec:results}, and forms the basis for the motivation for expansion of research into the energy impact of robotics software itself - as explained in section \ref{sec:discussion}.

For this study a total set of 683 potentially relevant studies were identified. After the application of the study design, as described in section \ref{sec:study_design}, the set of \textit{primary studies} consisted of 17 studies.
\section{Study design}
\label{sec:study_design}
This literature study has been designed and carried out by following 
well-accepted methodological guidelines on secondary studies
\cite{petersen2015guidelines_systematic, kitchenham2013systematic_review_guidelines, wohlin2012experimentation}.

This literature study targets multiple research questions. 
These questions will be given and motivated in subsection \ref{sec:study_design:research_questions}.
Then the search and selection of papers adhering to the inclusion and exclusion criteria began.
This process, and the criteria, are given in subsection \ref{sec:study_design:search_selection}.
After the selection was made final, hereon after called the \textit{primary studies}, the summarization and categorization process began. 
This process is the most important step for writing the literature study; the studies are made comparable by finding the commonalities and patterns in the field of study. 
This process is further explained in subsection \ref{sec:study_design:summ_categor}.


\subsection{Research Questions}
\label{sec:study_design:research_questions}
This study considers three research questions. These questions and their motivations are given in this subsection.

\vspace{5mm}

\textbf{[RQ1]} \textit{What are the publication trends of papers on energy efficiency in robotics software?}

\vspace{5mm}

To be able to judge any characteristics of the state-of-the-art of energy efficiency in robotics software, one needs to know the maturity of the field.
What publication trends are observed?

\vspace{5mm}

\textbf{[RQ2]} \textit{What is the state-of-the-art on analyzing and improving the energy efficiency in robotics software?}

\vspace{5mm}

This research question aims to answer what the state-of-the-art is for achieving and analyzing an increase of energy efficiency in robotics software.

\vspace{5mm}

\textbf{[RQ3]} \textit{What are the trade-offs when dealing with energy efficiency in robotics software?}

\vspace{5mm}

This research question aims to give insights into what Quality Attributes have been identified to trade-off with energy efficiency. 
It is valuable for researchers and practitioners to know that if one wants to increase energy efficiency, one can expect a decrease of some other attribute.

\subsection{Search and Selection}
\label{sec:study_design:search_selection}
\begin{figure}
    \centering
    \includegraphics[width=0.5\textwidth]{figures/selection_process_var2.png}
    \caption{The Search and Selection process}
    \label{fig:search_selec_process}
\end{figure}

The \textit{study design} is agreed and approved upon before starting the search and selection process. 
This is meant to prevent, as much as possible, any personal bias during search and selection, as the \textit{search string} and \textit{selection criteria} are already finalized.
This, and more threats to the validity of this report are detailed in section \ref{sec:threats}.
An overview of the search and selection process is given in figure \ref{fig:search_selec_process}.
The process, as displayed in the figure, is further elaborated on in this subsection.

\vspace{5mm}

\noindent\textbf{1. Initial Search:}
For the initial search, \textbf{Google Scholar}\footnote{\url{https://scholar.google.com/}} was used. 
Google Scholar is at the time of writing one of the largest and most complete database and indexing system for scientific literature.
It has been used as a data source for the following main reasons:
\begin{enumerate}
    \item The adoption of this indexer has proved to be a sound choice to identify the initial set 
    of literature studies for the snowballing process \cite{wohlin2014snowballing}.
    \item The query results can be automatically extracted from the indexer using Zotero\footnote{\url{https://www.zotero.org/}}.
\end{enumerate}

\begin{figure}
    \centering
    \textit{"(intitle:robot) AND (intitle:power OR intitle:green OR intitle:energy OR intitle:battery) AND software"}
    \caption{Search String}
    \label{fig:search_string}
\end{figure}

The results were retrieved using the \textbf{search string} as given in figure \ref{fig:search_string}. 
The search string is kept as general as possible so that potentially relevant studies that would be able to make it 
to the primary studies but might not match exactly, are not accidentally filtered out by the automatic search.

The search string consists of three main components, each is given and motivated below:
\begin{enumerate}
    \item \textit{intitle: robot} \newline
    Considering this literature study explicitly focuses on \textbf{robotics}; 
    the inclusion of \textit{robot} is warranted in the search string to retrieve studies in the context of robotics.

    \item \textit{intitle: power \textbf{OR} green \textbf{OR} energy} \newline 
    Considering this literature study explicitly focuses on \textbf{energy efficiency};
    the inclusion of \textit{power} \textbf{OR} \textit{green} \textbf{OR} \textit{energy} related titles is warranted in the search string to retrieve studies focussing on 
    these concepts.
    These concepts are deliberately logically seperated with an \textbf{OR} operator to prevent the exclusion of any studies that only focus on a subset.

    \item \textit{software} \newline
    Considering this literature study explicitly focuses on \textbf{software};
    the inclusion of software related papers is warranted.
    This is the only concept that is not explicitly limited to the \textbf{title}, as it is a broad, general term which might not always be mentioned
    explicitly in the title but does get mentioned in the abstract or keywords.
    
\end{enumerate}
The number of results at the time of performing the initial search were \textbf{683} potentially relevant studies.

\noindent\textbf{2. Filter Types and Duplicates}
During this step, all publication types that are not peer-reviewed by nature are filtered out. 
The potentially relevant studies that resulted from the initial search were thus automatically filtered to be only of any of these types: 
\textit{Journal Articles, Conference Papers, Book Sections}.
By filtering these types, the total number of potentially relevant studies went down to \textbf{615}. 
Then, the filtered collection was filtered automatically once more on 
syntactic (i.e. papers which are exactly the same) duplicates. 
Semantic (i.e. different papers about the same approach) duplicates were left to be filtered out manually during the data extraction phase, as explained in
\ref{sec:study_design:summ_categor}, in order to prevent unintentional removal. 
However, the approach as followed by this literature study is given for both below:

\vspace{1mm}

\begin{enumerate}
    \item[\textit{Syntactic}] In the case of a syntactic duplicate, only one record was kept. 
    These duplicates are in essence exactly the same, therefore which record to keep is not relevant.
    However, if applicable, the latest version (latest publication year) has been kept.

    \item[\textit{Semantic}] In the case of a semantic duplicate, meaning a paper was published in more than one instance 
    (for example, if a conference paper was extended to a journal version), only one instance has been counted as a primary study. 
    In those cases the journal version of the study has been preferred, as it is supposed to be the most complete; nevertheless, 
    both versions have been used in the data extraction phase and in the analysis of the publication trends (RQ1, see section \ref{sec:results}).

\end{enumerate}
After the duplicates were removed the total number of potentially relevant studies decreased to \textbf{610}.

\vspace{5mm}

\noindent\textbf{3. Application of Selection Criteria:}
During this step the \textbf{610} potentially relevant studies are filtered by applying the selection criteria. 
The study is added to the set of \textit{primary studies} in case it satisfies \textbf{all} of the inclusion criteria (\textit{i1-i5}) and \textbf{none} of the exclusion criteria (\textit{e1-e5}). 
These criteria consist of:
\begin{itemize}
    \item[i1] Studies focussing on robotics.
	\item[i2] Studies focussing on energy efficiency.
    \item[i3] Studies focussing on software aspects.
    \item[i4] Studies providing evaluation.
    \item[i5] Studies that are peer-reviewed.
    \item[i6] Studies written in English.
    
	\item[e1] Studies that, while focussing on energy efficiency, do not explicitly deal with any software aspect.
    \item[e2] Studies where energy efficiency is only used as an example.
    \item[e3] Secondary or Tertiary studies (literature reviews, theses etc).
    \item[e4] Studies that are not in the form of a Journal Article, Conference Paper, Book or Book Section.
    \item[e5] Studies not available as full-text.
\end{itemize}

The application of the selection criteria was done manually by following the steps given below. 
Each step was performed to see if any selection criteria could be decided based on the information gained by it. 
If a step decided one of the \textit{exclusion criteria} the next steps were not followed for that particular study, 
as it already warrants rejection. 
The following steps were followed for each of the \textbf{610} potentially relevant studies:
\begin{itemize}
	\item[S1] Read the \textit{Title}.
	\item[S2] \textit{Download} the study.
	\item[S3] Read the \textit{abstract}.
	\item[S4] Read the study \textit{full-text}.
\end{itemize}

Once the application of the selection criteria was completed for the entire set of potentially relevant studies,
a total of \textbf{15} papers were identified to satisfy \textbf{all} of the \textit{inclusion criteria} and
\textbf{none} of the \textit{exclusion criteria}. These formed the set of \textbf{considered studies}.

\vspace{5mm}

\noindent\textbf{4. Snowballing:}
% This table is not for snowballing, but does not want to move a page up if not put here.
% Even while using the [t] option.
\begin{table*}[t]
    \centering
    \caption{Data sheet columns.}
    \begin{tabular}{cllc}
        \toprule
            ID &
            Column Name & 
            Example Value & 
            Relevant RQ  \\
        \midrule
            1 &
               Date & 
                \textit{2020} & 
                RQ1 \\

            2 &
                Energy Metric & 
                \textit{FPS / W (Watt)} & 
                RQ2 \\

            3 &
                QA Trade-off & 
                \textit{Performance vs Efficiency} & 
                RQ3 \\
                
            4 &
                Application Domain & 
                \textit{Robot Exploration} & 
                RQ2 \\

            5 & 
                Identified Major Consumers & 
                \textit{Too many stops and turns in path} & 
                RQ2 \\

            6 & 
                Identified Improving Software Aspect & 
                \textit{Improved path finder} & 
                RQ2 \\
            
            7 &
                Major Contribution & 
                \textit{The actual improved, evaluated, path finder algorithm} & 
                RQ2 \\
            
            8 &
                Experiment & 
                \textit{None / Simulation / Real-World / Combination} & 
                RQ2 \\
            
            9 &
                Comparison Against State-Of-The-Art & 
                \textit{Yes / No} & 
                RQ2 \\
            
            10 &
                Energy Model &
                \textit{1 Unit of Distance = 1 Unit of Energy} & 
                RQ2 \\
        \bottomrule
    \end{tabular}
    \label{table:data_sheet}
\end{table*}
In this phase the automatic search was complemented with recursive \textit{backward} and \textit{forward} snowballing \cite{wohlin2014snowballing}.
During the \textit{backward} snowballing, all references of each \textit{considered study} were added to the potentially relevant studies. 
After each reference from each considered study was added, the steps for applying the selection criteria, as given above, were once more followed. 
After completing this iteration, each newly considered study was also used in the recursive backward snowballing process.

Following the backward snowballing, \textit{forward} snowballing was used. 
In this process each study that cites each considered study is added to the potentially relevant studies, hereafter each step for applying the selection criteria was once more followed, and the newly considered studies were recursively used in the forward snowballing process.

\vspace{5mm}

On completion of the snowballing process, the set of considered studies grew to \textbf{21} studies. These studies now form the \textit{primary studies} set.

\vspace{5mm}

\noindent\textbf{5. Exclusion during Data Extraction:}
Papers that made it to the collection of primary studies can still be removed from the set during the data extraction phase if they 
are found to satisfy one of the \textit{exclusion criteria} while reading the study full-text.

What the data extraction phase consists of, is explained in section \ref{sec:study_design:data_extract}.

For this literature study, a total of \textbf{4} papers have been excluded during data extraction. 

\subsection{Data Extraction}
\label{sec:study_design:data_extract}
During the data extraction phase, each \textit{primary study} is read full-text and its findings are used to construct a data sheet.
This data sheet aims to cover all similarities and patterns between the primary studies so that this literature study can be written, comparing their findings.
The data sheet was improved in multiple iterations, discussing each iteration with my supervisor.

The data sheet constructed during this phase; its columns, example values and their most relevant RQ are given in table \ref{table:data_sheet}.

\subsection{Data Synthesis}
\label{sec:study_design:data_synth}
Arriving at this step, the set of \textit{primary studies} is finalized, now consisting of \textbf{17} papers.
During the data synthesis phase, the filled out datasheet, as given in appendix \ref{appendix:data_sheet_1} and \ref{appendix:data_sheet_2}, 
was used to plot the results and gain insights.
These are given in section \ref{sec:results}.
\section{Results}
\label{sec:results}
As stated in the introduction, section \ref{sec:intro}, this literature study was complicated by 
the blurring lines between physical, digital, and biological spheres.
This literature study set out to discover the state-of-the-art of research on the energy efficiency impact of, \textbf{explicitly} \textit{robotics software}.
Meaning the explicit impact on energy efficiency of various aspects of the software itself.
It became quickly apparent during the search and selection process, as described in section \ref{sec:study_design:search_selection}, 
that few studies existed on this specific topic.
Out of all 683 potentially relevant studies, only two studies have been found that explicitly research this topic.

One study \cite{rahman2019cloud_robot_offloading} provides proof that a software architectural change; 
from on-board calculations to off-loading to more available robots or the cloud, actually impacts energy efficiency positively.
Another study \cite{hou2017novel_cloud_evaluation_model} presents a novel evaluation technique for cloud robotics, estimating energy efficiency.
The technique allows for identifying those robotics software aspects that consume relatively more energy. 
It can also be used to predict the energy consumption of a specific piece of robotics software, allowing it to be used during software development, 
to create more energy efficient software from the moment it is designed.

The fact that these two studies were the only ones explicitly covering research into the energy efficiency impact of robotics software aspects will
form the basis of the discussion, given in section \ref{sec:discussion}.
To prevent an insignificant literature study, the focus has been shifted from looking at software aspects, to seeing what impact 
robotics software in general can have on energy efficiency. The blurred distinction between software and hardware in robotics made the 
application of the inclusion criteria a tough process.
The final selection of primary studies is the result of a rigorous application of these criteria; 
a study had to explicitly cover \textit{some} software aspect in relation to energy efficiency.

In this section, the insights gained from the data sheet are summarized and given for each column in subsection \ref{sec:results:insights}.
Hereafter, this section is structured according to the research questions.
Each of those subsections gives a detailed explanation of the findings of this literature study in the context of that research question.
The main findings of each research question are presented at the end of each corresponding subsection.

% ========================================================================= Insights =========================================================================

\subsection{Data sheet insights}
\label{sec:results:insights}
The insights, as gained by each of the columns of the data sheet, are given in this subsection.
Any conclusions drawn from these insights in order to answer the research questions targeted by this literature study,
are given in their respective subsections \ref{sec:results:rq1_pub_trends}, \ref{sec:results:rq2_state_of_the_art} and \ref{sec:results:rq3_trade_off}.

\vspace{2mm}

\noindent\textbf{1. Date:}
\begin{figure}[t]
    \includegraphics[width=250pt]{figures/publication_trend_extended.png}
    \caption{Publication trends by type}
    \label{fig:pub_trends}
\end{figure}
In figure \ref{fig:pub_trends}, it can be observed that, even though this field of study has been around since before the change of the century,
it truly attained interest in the last decade (2010 - 2020), with a significant spike nearing 2020.
It should also be considered that the 1995 paper is only 3 pages long and discusses the energy impact
of, then novel, on-board motion calculation considering moving obstacles (i.e. people) \cite{barili1995efficient_motion}.
Quite advanced for those times, but nowadays a given.
It can also be observed that the more thoroughly peer reviewed, higher quality research; journal articles are only published in the last $\pm 5$ years.

\vspace{2mm}

\noindent\textbf{2. Metric:} % Might be more sensible to change to Efficiency Metric
% Various metrics used
% Power and Energy separate
% Joules / KiloJoules / NanoJoules most popular.
% Most intersting: FPS / W etc.
% Conclusion: Many different metrics, joules most popular, comparison difficult
It can be observed that the metrics used in the primary studies differ significantly.
In the case that a simplified energy model is used, as explained in part 10 of this subsection, the metric is the most unique;
units of distance traveled per units of energy \cite{mei2006mobile_exploration, patel2012exploration_strategy}.

All other metrics given, use a study specific metric in relation to either; energy, \textbf{Joules} (\textit{J}) or power, \textbf{Watts} (\textit{W}).
The three most common, descriptive and comparable metrics observed are:
\begin{enumerate}
    \item $FPS\footnote{Frame rate (expressed in frames per second or FPS)} / Watt$ \cite{cheng2018FPGA_image_recognition}
    \item $Joules / Meter$ \cite{licea2013wireless_comms}
    \item $Joules / H$ or $Watts / H$ \cite{kim2016firefighting_robot,barili1995efficient_motion}
\end{enumerate}

\vspace{2mm}

\noindent\textbf{3. QA Trade-off:}
% Various trade-offs mentioned.
% Used in RQ 3.
From the 17 primary studies, 15 present a technique to significantly improve energy efficiency.
In each of these studies, a trade-off between the improved energy efficiency and some other attribute(s) of the system can be observed. 
The remaining two studies present techniques to analyze the energy efficiency of a system, and do not present any trade-off with any other
attribute of a system.

\vspace{2mm}

\noindent Each trade-off has been formally mapped to the \textit{Software Quality Attributes\cite{iso2011quality_attributes}}.
However, considering robotics, there are attributes of a robotic system that do not exist in software.
One such attribute has been found to trade-off with Energy Efficiency; Response.

Response is an activity undertaken by the robotic system at the arrival of stimulus \cite{shakhimardanov2007robotic_response}.
Response is an attribute paramount to robotics; an intelligent, fast and adequate response to stimulus is to be desired.

From the primary studies it can be observed that Energy Efficiency trades-off with the following - official - attributes:

\begin{itemize}
    \item Performance Efficiency
    \item Maintainability
    \item Reliability
\end{itemize}

These trade-offs are further detailed; answering the third research question, in subsection \ref{sec:results:rq3_trade_off}.
Considering Response, as it is not an official quality attribute as described in the referenced literature, it will be discussed informally.

\vspace{2mm}

\noindent\textbf{4. Application Domain:}
% Exploration / Service robots
\begin{figure}[t]
    \includegraphics[width=250pt]{figures/domain_freq_barplot.png}
    \caption{Application domains frequency}
    \label{fig:app_domains}
\end{figure}
In figure \ref{fig:app_domains} it can be observed that the application domain of \textbf{Robot Exploration} is the most frequent
domain considered in the primary studies. 
The domain of \textbf{Industrial Robot} is in second place with less than half the frequency.

This is interesting, considering that in terms of emmissions and energy saving potential, there is far more to achieve in the industrial domain
as specified in the introduction; section \ref{sec:intro}.
From this statistic one could thus reason that the combination of functional and efficiency improvements form the main motivation for improving energy efficiency. 
The extension of the operational time of a mobile robot, as a result of the improved energy efficiency, 
is far more functionally relevant than reducing the energy consumption of an industrial robot;
considering the industrial robot has a continuous power supply, energy efficiency is not as functionally important as it will
not deliver any functional improvement.
On the contrary, one could in such a case reason that any improvement of Energy Efficiency decreases Performance Efficiency.
This will be further detailed in subsection \ref{sec:results:rq3_trade_off}.

\vspace{2mm}

\noindent\textbf{5. Identified Major Consumers:}
% Software inefficiencies
% Hardware inefficiencies
% Hardly win-win, only when time reduced, hardware inefficiency reduced etc. (only with software?)
The identified major consumers in the primary studies vary considerably, as can be seen in appendix \ref{appendix:data_sheet_1} and \ref{appendix:data_sheet_2}.
This is a result from the different application domains and their implementation specific details.
However, a common theme can be identified:

No matter the application domain, any identified inefficiency can always be traced back to a (physical) \textit{hardware} inefficiency, 
which is solved by improving \textit{software}.

Considering robots use software to control their hardware, and the fact that robots exist to satisfy some physical need, this is logical.
Within this domain of physical inefficiencies, groups of identified major consumers can be observed.
The observation and formulation of these groups are, of course, subject to personal bias.
This threat to validity has however been mitigated, more about this can be read in section \ref{sec:threats}.
The observed groups consist of the following:

\vspace{2mm} \noindent \textbf{5.1. Movement related:}
Many papers consider the significant energy consumption of robotic movement. 
Especially stops, directional changes, the degree to which the direction is changed and acceleration and decelleration are commonly observed
to contribute significantly to the total energy consumption 
\cite{mei2005energy_consumers_identified, mei2006mobile_exploration, xie2018mecanum_wheel,kaitwanidvilai2020industrial_robot_cycle_time}.

\vspace{2mm} \noindent \textbf{5.2. Inefficient Hardware:}
Inefficient hardware is commonly identified to be a major consumer. 
This category encapsulates those circumstances where the hardware used is inefficient compared to other solutions 
(e.g. an accelerator for vision-based tracking instead of a more efficient FPGA\footnote{Field Programmable Gate Array} \cite{cheng2018FPGA_image_recognition})
or the hardware is used inefficiently (e.g. loss of traction due to a too heavy payload weight \cite{kim2016firefighting_robot}).

\vspace{2mm} \noindent \textbf{5.3. Inefficient Procedure:}
This group seems to be encapsulating the previous two at first glance. 
However, considering the specificity with which papers focus on one aspect of a system such a paper typically does not take the encapsulating procedure into account.
Using the Oxford definition of a procedure\footnote{\url{https://www.lexico.com/en/definition/procedure}}; a series of actions conducted in a certain order or manner.
This group thus considers identified major consumers where the set of actions conducted in the specified manner would result in an inefficient use of energy.

Some examples would consist of the following:
\begin{itemize}
    \item The use of a traditional DWA local trajectory planner in combination with mecanum robotic wheels \cite{adascalitei2011mecanum_wheels}.
    \item An industrial multi-robot production cell using an inefficient scheduler, resulting in significant idle times \cite{wingstrom2013robot_cell_scheduling}.
    \item Broadcasting data over a wireless connection from a bad position, resulting in a significantly low channel gain and a lossy connection \cite{licea2013wireless_comms}.
\end{itemize}

These examples could all be improved using an improved, or in some cases a completely different, procedure.

\vspace{2mm} \noindent \textbf{Software as consumer:}
From all the primary studies, only one solely identifies software itself as the main consumer \cite{hou2017novel_cloud_evaluation_model}.
It presents a novel cloud evaluation method; which evaluates the energy efficiency of the software itself.
Using the method, the execution of the software itself can be made more efficient. 
The fact that this is the only study addressing this aspect of robotics software, forms the basis of the discussion in section \ref{sec:discussion}.

\vspace{2mm}

\noindent\textbf{6. Identified Improving Software Aspect:}
% Mostly solving hardware with software
Among the primary studies, the identified major consumers are bettered by improving the robotics software;
as this literature study explicitly targets such studies.
As the presented improvements are implementation specific, they are hardly comparable.
However, a common theme among the improvements can be identified:
each improvement solves a hardware inefficiency using software; however, each improvement does so to a varying degree. 

Groups can be observed among the identified improving aspects.
As the improving aspects are cohering with the observed major consumers, as they aim to improve them,
the groups are inherently related.
The observed groups consist of the following:

\vspace{2mm} \noindent \textbf{6.1. Improved Hardware:}
Those systems and applications where the used hardware can be considered inefficient, 
either in its use for that specific application (i.e. the accelerator vs FPGA),
or its use in general (i.e. the loss of traction due to a too heavy payload weight)
can be improved in terms of energy efficiency by improving the used hardware.
Some examples of improving the system's hardware, consequently improving its energy efficiency consist of:
\begin{itemize}
    \item Off-loading computations from on-board hardware to more energy optimized, better utilized cloud infrastructure \cite{rahman2019cloud_robot_offloading}.
    \item The use of an FPGA for vision-based tracking instead of the common accelerator \cite{cheng2018FPGA_image_recognition}.
    \item Adding sub-robots to a system to distribute payload weight and prevent energy inefficient use of hardware \cite{kim2016firefighting_robot}.
\end{itemize}

For each of these examples, as observed in the primary studies, the betterment of energy efficiency has been proven, evaluated and tested
as the selection criteria for making it into the primary study set demands so.

\vspace{2mm} \noindent \textbf{6.2. Improved Procedure:}
As described before, the procedure - a series of actions conducted in a certain order or manner - can be a source of inefficiency in terms of energy.
Therefore improving these procedures can lead to conducting these actions more energy efficiently.
Some examples of improving procedures to improve energy efficiency consist of:
\begin{itemize}
    \item Improving algorithms related to robotic motion to reduce stops, turns and moments of acceleration or decelleration by improving either the path planning itself
    or the obstacle avoidance algorithm. Resulting in more energy efficient motion 
    \cite{xie2018mecanum_wheel,mei2006mobile_exploration, mei2005energy_consumers_identified, barili1995efficient_motion, jia2004grid_strategy_exploration, kaitwanidvilai2020industrial_robot_cycle_time}.
    \item Reducing the idle time of robotic systems by improving the procedures responsible for task distribution and execution 
    \cite{kaitwanidvilai2020industrial_robot_cycle_time, gurel2019industrial_robot_scheduling, wingstrom2013robot_cell_scheduling}.
    \item Making a robotic system able to predict the energy cost of stimuli and base the decision to act on it on the available energy budget \cite{kirtay2013humanoid_emotion}.
\end{itemize}

It should be noted that although a clear relation can be seen between the groups observed for major consumers and improving aspects, 
a movement related specific group is missing for identified improved aspects.
The simple reason that this is the case, is the fact that improving movement related aspects inherently requires either 
an improved procedure for planning that movement, improved hardware that consumes less energy for the same amount of movement
or a combination thereof.

\vspace{2mm}

\noindent\textbf{7. Major Contribution:}
% For example the entire system, integrating the software aspects to improve inefficiency.
For each of the primary studies the major contribution is the implementation and evaluation of the identified software aspect that would
improve the energy inefficiency. 
This column can therefore be seen as an extension of the previous column; part 6 of this subsection.
These contributions are further detailed; answering the second research question, in subsection \ref{sec:results:rq2_state_of_the_art}.
Grouping these contributions would largely overlap with the groupings for the previous two grouped metrics, and is therefore a redundant activity.

\vspace{2mm}

\noindent\textbf{8. Experiment:}
% Experiment performed most common: simulation, why?
\begin{figure}
    \includegraphics[width=250pt]{figures/exp_freq_barplot.png}
    \caption{Experiment distribution}
    \label{fig:experiment_distr}
\end{figure}
From figure \ref{fig:experiment_distr}, it can be observed that most studies performed an experiment, only one primary study did not.
It can be observed that most studies, 9 out of 17, perform an experiment in the form of a simulation, 
this is without counting the combination of real-world and simulation.
Considering those; 12 out of 17 primary studies used a simulation as an experiment.
Out of the primary studies, 4 studies performed a real-world experiment and 3 performed a combination of real-world and simulated experiments.

\vspace{2mm}

\noindent\textbf{9. Comparison Against State-Of-The-Art:}
% Mostly yes, important for validity!
To verify the validity of the claims presented in the primary studies, the comparison against the state-of-the-art is important.
In 14 of the 17 primary studies a comparison was made.
From the three studies that had no comparison, one was a paper presenting a novel cloud evaluation method \cite{hou2017novel_cloud_evaluation_model}
which has no state-of-the-art to compare to as it presents something novel. 
It can thus be safely concluded that 15 of the 17 primary studies uphold the comparison.
This is of importance as this literature study's validity is dependent upon the validity of the literature it is based on.

The most notable comparisons against the state-of-the-art, as observed in the literature, consist of the following.
The specific results compared to the state-of-the-art for each paper are 
purposefully omitted to prevent any erroneous comparisons between the papers and their results.
Each of the results should be considerd on their own, 
compared to the relative state-of-the-art, and for such nuance, 
one should read the referenced literature.

\vspace{2mm} \noindent \textbf{9.1. Exploration algorithm.}
The state-of-the-art with respect to exploration algorithms for mobile robots
consisted of selecting the next target for exploration based on the utility - amount of area to be explored - 
and the cost - the distance of the target to the current position - of the frontier cell 
\cite{burgard2005multi_robot_exploration, simmons2000multi_robot_exploration,zlot2002multi_robot_exploration}.
In the studied literature it is however proven that if the next target is selected based on the orientation of the robot,
overlap in the robot trajectory is guaranteed to be impossible \cite{mei2006mobile_exploration}.
Reducing energy consumption, and reducing total travel distance; enabling the mobile robot 
to finish exploring the area faster while consuming less energy. 

\vspace{2mm} \noindent \textbf{9.2. Energy efficient macanum motion control.}
Mecanum wheels are the state-of-the-art in industry when it comes to omnidirectional wheel designs with a high load capacity \cite{xie2018mecanum_wheel}.
The Ilon Mecanum wheel is one of the practical omnidirectional wheel designs utilized in the industry, 
and it has the advantage of high load capacity over other omnidirectional wheel designs \cite{adascalitei2011mecanum_wheels}.
However, the mecanum wheel trades off maneuverability against motion efficiency, and its inefficient energy usage for generating omnidirectional 
motions increases the energy consumption of the robot \cite{diegel2002improved_mecanum_wheel}.
The study extends the popular, well-known, robot operating system (ROS) dynamic window approach (DWA),
a reactive collision avoidance navigation concept \cite{fox1997dwa_paper}.
DWA has proven performance over a range of extended applications \cite{brock1999dwa_usage_1,ogren2005dwa_usage_2,kiss2012dwa_usage_3}.
The study added energy-related criterion to the DWA local trajectory planner,
which proved to minimize energy consumption per dynamic window.

\vspace{2mm} \noindent \textbf{9.3. Energy efficient, multi-robot firefighting}
To lessen the deaths \cite{fahy2015firefighter_deaths} and property damages \cite{karter2013fire_damages} 
caused by fires in the United States, many researchers have studied various kinds of solutions. 
One of the most effective solutions is using a firefighting robot that can 
substitute for firefighters on the fire scene. This state-of-the-art solution
is however often implemented as a single robot (e.g. FIRO-M), 
carrying a heavy hose full of water into the fire scene.
The robot often suffers from a loss of traction because the hose full of water
is too heavy, or the hose gets stuck on a corner the robot has just driven around \cite{kim2016firefighting_robot}.
To get unstuck, the robot will have to drive back and forth. 
Both consume significant amounts of energy that is not used for fighting the actual fire.
To address this, the study presents MRESS - multi-robot energy saving system - to increase
the operating time of a firefighting robot.
The addition of sub-robots, aiding in carrying the hose, 
allow for a better weight distribution, preventing a loss of traction, 
and the advanced vision-based fire hose tracking method prevents 
that the hose gets stuck on corners.
Compared to the state-of-the-art, the single firefighting robot, 
this saves significant amounts of energy and allows for a longer operating time.

\vspace{2mm}

\noindent\textbf{10. Energy Model:}
% Interesting for future research / comparison between the relative studies
The energy model used is given for each of the primary studies in appendix \ref{appendix:data_sheet_2}.
The validity of the contributions of the studies is dependent on the validity of the energy model used (if applicable), 
these are therefore recorded. 
They can also be valuable for any practitioner or researcher that seeks insight into energy models for robotics simulation.

Many different energy models have been used by the various primary studies, some try to simulate the energy consumption 
as precisely as possible by simulating drag, torque, acceleration, decelleration, etc. with extensive mathematics and physics equations. 
These models are called \textit{representational}.

Opposite from representational models, is the \textit{abstract} model. 
This model is used in cases where representational simulation is not necessary, 
and only the relevant difference in energy consumption between that which is researched is of importance.
These models can therefore be simple in their nature.

\vspace{2mm} \noindent \textbf{Energy model shortcoming:}
An interesting observation can be made from studying the various energy models present in the studied literature; 
none of the energy models are capable of simulating any computational energy usage.
Therefore, in studies where this is of importance, only a real-world experiment was able to provide these insights.

\vspace{2mm} \noindent \textbf{10.1. Abstract Simple Model}
The abstract simple model is relevant in use cases where only differences in relevance to one another of that which is researched is of importance.
The gathered metrics will be in no way representational and comparable to the real world, 
but can be used to simply and clearly depict differences with relative credibility.
Espcially when coupled with results from a real-world experiment, as can be often seen in studies that use such models.

An example of an abstract simple model would consist of the following:

Two papers use the same approach as their energy model, which can be identified as an abstract simple model.
It is the only model among the primary studies which is used by multiple papers;
these papers are written by different authors but within the same field of study; mobile robot exploration.
The model they use consists of a simulated grid, where each grid cell consists of 1x1 Units of Distance, 
and travelling 1 Unit of Distance equals 1 Unit of Energy.
Each stop costs 0.5 Units of Energy and each 45° turn costs 0.4 Units of Energy, each additional 45° adding another 
0.2 Units of Energy to the total cost.
Meaning: a 90° turn would cost 0.6 Units of Energy and a 135° turn would cost 0.8 Units of Energy etc.

\vspace{2mm} \noindent \textbf{10.2. Representational Complex Model}
Representational energy models are models that can be used in simulations where results would need to be comparable to the real-world.
These models are often complex of nature and require extensive information about the robotic system considered, such as
friction coefficients (e.g. rolling, sliding and viscous), wheel radius, robot mass, static friction torgue, 
idling power consumption, gear ratio etc.
This information is then to be used in accurate physics equations that are capable of modeling the robotic system accurately and
representational to the real-world.

Mathematics and kinematics\footnote{\url{https://www.lexico.com/en/definition/kinematics}} 
- the branch of mechanics concerned with the motion of objects without reference to the forces which cause the motion -
are extensively used in these models to get results as close to the real-world as possible.

% ========================================================================= RQ 1 =========================================================================

\subsection{Results - publication trends (RQ1)}
\label{sec:results:rq1_pub_trends}

In this section the results obtained when analyzing the publication trends on energy efficiency in robotics software are presented.
Understanding the publication trends in the field of study is essential for interpreting the results of this literature study as it gives
an idea of the maturity of the field. 
From these findings, as stated in part 1 of subsection \ref{sec:results:insights}, we can conclude that the maturity of the field is rather limited,
considering the number of publications decrease significantly the further back we go in time.

It is important to take this into account for the findings of this literature study, presented in the following subsections 
\ref{sec:results:rq2_state_of_the_art} and \ref{sec:results:rq3_trade_off}, and for the discussion as presented in section \ref{sec:discussion}.
Considering the aforementioned difficulty with the initial goal of this literature study; from the publication trends we can see that this 
is probably the case because of a rather immature field of study.

\vspace{2mm}

\noindent\fcolorbox{black}[HTML]{FFFFFF}{\parbox{0.47\textwidth}{%
\noindent \textbf{Main Findings.}
\begin{enumerate}[nolistsep]
\item The field of study has been around since before the change of the century.
\item The field can still be considerd immature as publications only recently attained in numbers.
\end{enumerate}}}

% ========================================================================= RQ 2 =========================================================================

\subsection{Results - state-of-the-art (RQ2)}
\label{sec:results:rq2_state_of_the_art}
In this section the state-of-the-art in analyzing and improving energy efficiency in robotics software is presented as found from studying the primary studies.

\vspace{2mm}

\noindent\textbf{1. Analyzing:}
The novel cloud evaluation method \cite{hou2017novel_cloud_evaluation_model}, introduces a whole new paradigm in robotics software.
This can be concluded as it is the only paper out of 683 potentially relevant studies, and 17 primary studies, 
that explicitly looks at the execution of robotics software itself for improvement of energy efficiency.

It presents a novel method to evaluate the energy efficiency of a specific piece of software.
The method can be used on existing software, to identifiy bottlenecks, or used during the development of the software itself;
providing the ability to improve the energy efficiency of the software execution during design time.

It can be considered the state-of-the-art of evaluation and analysis, in robotics software.
However, it should be considered that this was the only paper that focussed so explicitly on software and its impact on energy.
Considering the 16 other primary studies, and the observations made as stated in part 8 of subsection \ref{sec:results:insights}; 
it can be stated that another big standard in this field is the evaluation through practice; 
in the form of a simulation, a real-world experiment or a combination thereof.

\vspace{2mm}

\noindent\textbf{2. Improving:}
The techniques to improve energy efficiency in robotics software vary significantly across the primary studies. 
As stated in part 5 and 6 of subsection \ref{sec:results:insights}; a trend observed is that they mostly solve hardware (physical) 
inefficiencies using software solutions, like an improved algorithm.

As stated before, this literature study initially set out to research the state-of-the-art in terms of improving the
energy efficiency of robotics software \textit{itself}.
The fact that only two of the primary studies presented such research forms the basis of the discussion in section \ref{sec:discussion}. 
This section will therefore detail what has been found by studying the primary studies in the context of improving energy efficiency 
by \textit{using} robotics software.
Each primary study presented an evaluated software solution that improves energy efficiency.
The techniques extracted from the primary studies consist of:

\vspace{1mm}

\textbf{1 Off-loading computations.} The technique to off-load computations to other, nearby, robots that are more 'available' 
(i.e. robots that have more resources available for such computations relative to the current one), or to off-load it to the cloud.
The concept here is that the cloud infrastructure (hardware itself, hardware utilization, etc) is more energy-optimized compared to the
hardware used on the robots themselves, and will thus result in an improved energy efficiency.
Even though some energy is wasted in the transmission of data, the overall energy consumption is decreased \cite{rahman2019cloud_robot_offloading}.
    
\vspace{1mm}

\textbf{2 Advanced motion algorithm.} The technique to improve the motion algorithm, such as path finding algorithms for mobile robot exploration. 
The improvement involves incorporating techniques to facilitate energy efficient movement.
Techniques like \textbf{3, 4 and 5} can be implemented in an advanced motion algorithm.
Like implementing a more advanced obstace-avoidance algorithm, enabling the robot to steer less by steering earlier to avoid the obstacle \cite{xie2018mecanum_wheel}.
Also, many existing studies select the next target based on the utility and cost of the frontier cells 
\cite{burgard2005multi_robot_exploration, simmons2000multi_robot_exploration,zlot2002multi_robot_exploration} 
However, study \cite{mei2006mobile_exploration} proves that if the next target is selected based on the orientation of the robot, 
overlap in the robot trajectory is guaranteed to be impossible; reducing energy consumption and reducing the total travel distance.
Enabling the mobile robot to finish exploring the area faster while consuming less energy.
    
\vspace{1mm}

\textbf{3 Limiting motion changes.} The technique to limit stops, directional changes (turns) and the degree to which the direction is changed as much as possible 
significantly improves energy efficiency. By the very nature of this technique, an improved obstacle detection and avoidance algorithm
is needed for mobile robotic systems
\cite{xie2018mecanum_wheel, kim2016firefighting_robot, benkrid2016multi_robot_exploration, barili1995efficient_motion, 
jia2004grid_strategy_exploration, mei2005energy_consumers_identified, patel2012exploration_strategy}.
    
\vspace{1mm}

\textbf{4 Limiting motion speed.} The technique to limit motion at high speeds, with numerous moments of acceleration and decelleration
\cite{wingstrom2013robot_cell_scheduling}.
    
\vspace{1mm}

\textbf{5 Limiting idle time.} The technique to prevent idle time as much as possible \cite{gurel2019industrial_robot_scheduling, 
kaitwanidvilai2020industrial_robot_cycle_time, wingstrom2013robot_cell_scheduling}.
    
\vspace{1mm}

\textbf{6 Limiting unnecessary communication.} The technique to limit data transmission by preventing the transmission of duplicate 
or otherwise unnecessary data in multi-robotic systems \cite{huh2013distributed_swarm}.

\vspace{1mm}

\textbf{7 Limiting physical inefficiencies.} The technique to limit physical inefficiencies (e.g. loss of traction because of payload weight), if possible, by adding more robots to the system which would cause the overall
energy consumption to go down as the loss of traction was consuming more energy than the addition of the subrobots \cite{kim2016firefighting_robot}.
    
\vspace{1mm}

\textbf{8 Using optimized hardware.} The technique to use more advanced hardware (i.e. more energy-optimized, desktop grade, hardware instead of custom robotic hardware) 
on robots in combination with energy-optimized software \cite{cheng2018FPGA_image_recognition}.
    
\vspace{1mm}

\textbf{9 Limiting transmission loss.} The technique that sacrificing some energy on finding a better position for the transmission of data over a wireless connection,
to increase higher channel gain, will ultimately improve energy efficiency as less time is spend and wasted on (re)transmitting 
data over a lossy wireless connection \cite{licea2013wireless_comms}.

\vspace{1mm}

\textbf{10 Predicting task's energy cost.} The technique to be able to predict the energy cost of any stimuli and reject said stimuli if it is predicted to exceed some energy threshold 
\cite{kirtay2013humanoid_emotion}.

\vspace{2mm}

\noindent\fcolorbox{black}[HTML]{FFFFFF}{\parbox{0.47\textwidth}{%
\noindent \textbf{Main Findings.}
\begin{enumerate}[nolistsep]
\item The state-of-the-art on analyzing the energy efficiency of robotics software consists of evaluation techniques capable of
estimating the energy efficiency of a piece of robotics software and identifying energy efficiency bottlenecks in existing robotics software.
\item The most common way to analyze the energy efficiency of robotics software consists of performing experiments, evaluating the results.
\item The state-of-the-art on improving the energy efficiency consists of:
    \begin{enumerate}
        \item Off-loading computations to more energy-optimized infrastructure.
        \item Improved path finding, obstacle avoidance etc.
        \item Limit physical inefficiencies, idle time, acceleration, decelleration, stops, turns, directional changes and the extent of the directional change.
        \item The use of more energy-optimized hardware on robotics themselves.
        \item Sacrificing some energy to achieve higher efficiency (e.g. finding a better location with better signal for data transmission).
        \item Using elaborate software to be able to predict the energy cost of stimuli and reject if necessary.
    \end{enumerate}
\end{enumerate}}}

% ========================================================================= RQ 3 =========================================================================

\subsection{Results - QA trade-off (RQ3)}
\label{sec:results:rq3_trade_off}
From the primary studies, it can be observed that most techniques that significantly improve energy efficiency come with a cost to some other attribute of the system.
As stated in part 3 of the subsection \ref{sec:results:insights}, the attributes have been mapped to \textit{Software Quality Attributes\cite{iso2011quality_attributes}}
if possible.

In figure \ref{fig:trade_off_freq} it can be observed that the \textbf{Performance Efficiency} is the most common QA trade-off with 8 occurrences in the 17 primary studies.

\vspace{1mm}

- \textbf{Performance Efficiency} trades-off with Energy Effiency in any case where physical or software inefficiencies cannot be improved or do not exist. 
Techniques that are similar to techniques \textbf{3, 4, 5 and 9} from subsection \ref{sec:results:rq2_state_of_the_art}, 
will inherently decrease Performance Efficiency as it is the only way such techniques improve Energy Efficiency.

It should be noted that some techniques, similar to techniques \textbf{1, 2, 6, 7 and 8}, both improve the Performance Efficiency and Energy Efficiency. 
It has been observed that this is often the case using techniques that will improve a given physical or software inefficiency. 
For example, technique \textbf{2} improves the path planning algorithm, guaranteeing \textit{no path overlap} during area exploration. 
It has been proven that this not only improves Energy Efficiency, but also allows the robot to explore the unknown area faster; 
thus improving Performance Efficiency \cite{mei2006mobile_exploration}.

\vspace{1mm}

\begin{figure}[t]
    \includegraphics[width=250pt]{figures/tradeoff_freq_barplot.png}
    \caption{QA Trade-off frequency distribution}
    \label{fig:trade_off_freq}
\end{figure}

- \textbf{Maintainability} trades-off with Energy Efficiency in any case where improving Energy Efficiency requires more elaborate and complex software and/or hardware. 
This is the case for each of the observed techniques, to varying degree, as presented in subsection \ref{sec:results:rq2_state_of_the_art}; \textbf{1 - 10}.

\vspace{1mm}

- \textbf{Reliability} trades-off with Energy Efficiency in any case where improving Energy Efficiency requires the system to 
stop operating for some criteria, like using techniques similar to \textbf{10}, or where the chance of operational failure is increased, 
like increasing the possible points of failure during operation using techniques similar to \textbf{1, 7 and 9}. 

Besides the fact that certain QAs need to be traded-off in order to improve energy efficiency, the extent to which this is required 
matters just as much, if not more.
For each system, the hit to the traded-off QA and the improvement of Energy Efficiency will be significantly different.
Thus, an indication of expected results cannot be given.
However, one could reason that a disproportional hit to the traded-off QA, relative to the increase in Energy Efficiency might not be worthwile.

Study \cite{kaitwanidvilai2020industrial_robot_cycle_time} for example states: 

\noindent\textit{"This method reduces energy consumption from 8155.20 to 7148.6 J, 
a decrease of 12.3\%. On the other hand, the total moving time is increased by 71.8\% from 6.60 to 11.34 s".}

In case the system in question can suffer a 71.8\% reduction in Performance Efficiency for a 12.3\% increase in Energy Efficiency, it might be worthwile.
However, it can be considered a good example of a trade-off which might not be worthwile for most systems, let alone time-critical systems.

\vspace{1mm}

\noindent \textbf{Informal attribute discussion}
\newline
\textbf{Response} is a quality attribute paramount to robotic systems, 
as robotic systems result from a need to manipulate some environment using hardware, controlled by software.
Response is an activity undertaken by the robotic system at the arrival of stimulus \cite{shakhimardanov2007robotic_response}.
A robotic system's Response to stimulus is desired to be intelligent, fast and adequate.
However, Response is not referenced as a quality attribute in the official literature and is thus discussed informally.

Response trades-off with Energy Efficiency in any case where improving Energy Efficiency requires the system to reduce its responsiveness to 
the environment it exists in and any stimulus resulting from it. 
This trade-off can be observed when implementing techniques where the robot has to limit its movement; techniques similar to \textbf{3 and 4}.

\vspace{2mm}

\noindent\fcolorbox{black}[HTML]{FFFFFF}{\parbox{0.47\textwidth}{%
\noindent \textbf{Main Findings.}
\begin{enumerate}[nolistsep]
\item The most common QA trade-off in order to improve Energy Efficiency is \textbf{Performance Efficiency}.
\item Some techniques will improve both Energy Efficiency and \textbf{Performance Efficiency} at the same time, 
these techniques mostly consist of techniques solving physical and/or software inefficiencies.
Like techniques similar to \textbf{1, 2, 6, 7 and 8}.
\item Each QA trade-off will have a varying degree of impact for each specific system. 
It is up to the reader to decide if such trade-offs are worth it for their system.
\end{enumerate}}}
 
\section{Discussion}
\label{sec:discussion}
This literature study initially set out to research the state-of-the-art in terms of improving the
energy efficiency of robotics software \textit{itself}.
The fact that only two of the primary studies presented such research forms the basis of this discussion.

\vspace{2mm}

Their publication dates are 2017 \cite{hou2017novel_cloud_evaluation_model} and 2019 \cite{rahman2019cloud_robot_offloading}, 
indicating the infancy of the field of study. The importance of such research is significant, 
considering the high emmisions and use of energy in the automation sector, as explained in section \ref{sec:intro}.

\vspace{2mm}

This literature study has mainly studied papers improving the energy efficiency of existing inefficiencies, 
be it the loss of traction (physical) \cite{kim2016firefighting_robot}, 
the use of an FPGA\footnote{Field Programmable Gate Array} in combination with a neural network
to improve the acceleration on-board (hardware) \cite{cheng2018FPGA_image_recognition} 
or the improvement of an inefficient path finder (software) \cite{mei2006mobile_exploration}.

\vspace{2mm}

One could thus say with confidence that, while there is still much to gain from that field of study,
the impact of robotics software itself on energy efficiency is not yet sufficiently researched.
As our society puts a growing effort towards a sustainable future, it warrants the expansion of research into this field of study.
The biggest improvement in energy efficiency is going to be achieved by combining all the aforementioned state-of-the-art methods for improving 
energy efficiency in section \ref{sec:results:rq2_state_of_the_art}, with software that is designed to be as energy efficient as possible.
A good start has already been provided with a significant contribution; 
a novel green evaluation method for cloud robotics \cite{hou2017novel_cloud_evaluation_model}.
Which enables the identification of bottlenecks in robotics software in terms of energy efficiency, 
and the prediction of the energy efficiency of a piece of software during development.
However, the fact that this is the only study, out of 17 primary studies, selected from 683 potentially relevant studies, 
which explicitly focuses on the identification of energy inefficient software, should warrant motivation for
the expansion of such research.
\section{Threats To Validity}\label{sec:threats}
\subsection{Internal Validity}
This threat has been mitigated as much as possible by defining the research protocol, as explained in section \ref{sec:study_design}, 
as rigorously as possible.
Iteratively defining it by discussing it after each iteration with my supervisor.

\subsection{External Validity}
The most severe potential external threat to the validity of this literature study is that the primary studies would not 
be representative of the state-of-the-art on energy efficiency in robotics software.

To avoid this to happen, the search strategy applied consisted of both automatic search and backward and forward snowballing 
\cite{wohlin2014snowballing}. Specifically, the presence of potential gaps left out by the automatic search was mitigated by 
the snowballing technique. 

This enlarged the set of relevant studies by considering each study selected in the automatic search, 
and focussing on those papers either citing or cited by it. 
Also, only peer-reviewed papers were considered and secondary and tertiary studies exlcuded.
This potential bias did not significantly impact this literature study, since the considered papers have undergone a rigorous
peer-review process, which is a well-established requirement for high quality publications. 

The inclusion and exclusion criteria were also rigorously and iteratively defined, again discussing it with my supervisor,
before the study design was put into action.

\subsection{Construct Validity}
This potential bias was mitigated by automatically searching the studies on any data source as indexed by \textbf{Google Scholar}.
The search string was also kept as general as possible, as explained before in section \ref{sec:study_design:search_selection}.

\subsection{Conclusion Validity}
Potential biases during the data extraction process were mitigated by rigorously and iteratively defining the data sheet with my superivsor.
By doing so, the alignment of the data extraction process with the research questions was guaranteed.
Furthermore, any potential threats to conclusion validity were mitigated, in general, by applying the best practices on systematic
literature reviews in each phase of this study, as stated in 
\cite{petersen2015guidelines_systematic, kitchenham2013systematic_review_guidelines, wohlin2012experimentation}.

This makes this literature study easy to be checked and replicated by other researchers.
\section{Conclusions}\label{sec:conclusions}
To conclude this literature study; the waste of energy by persisting inefficiencies in robotic hardware, software or the physical world
(e.g. the weight of payload) can be significantly improved by applying the findings of this literature study as set out in section 
\ref{sec:results}, specifically subsection \ref{sec:results:rq2_state_of_the_art}.
Despite the improvements in CO2 emmissions and energy consumption over the past years, as mentioned in section \ref{sec:intro}, 
Fysikopoulos et al. \cite{fysikopoulos2012automotive_energy_consumption} assert that 20\% to 40\% unnecessary use of energy may 
still be found in industrial firms.

\vspace{5mm}

The unnecessary use of energy can only be significantly reduced by combining the tactics for solving known inefficiencies in the field of robotics,
as this literature study has studied and presented as its main findings, with the future research into the impact on energy efficiency by the
robotics software itself.
The motivation for the expansion into this field of study has been given in section \ref{sec:discussion} and is the most important
contribution of this literature study. 

\bibliographystyle{IEEEtran}
\bibliography{references}

\clearpage
\appendices

\section{Primary Studies}
\label{appendix:primary_studies}

\begin{table}[h]
    \centering
    \caption{Collection of primary studies.}
    \begin{tabular}{lp{5cm}p{9cm}c}
        \toprule
            {ID}
                {Publication Type} & 
                {Authors} & 
                {Title} &
                {Date} \\
        \midrule
            {1.}
                {conferencePaper} &
                {Mei, Yongguo; Lu, Yung-Hsiang; Lee, CS George; Hu, Y. Charlie} &
                {Energy-efficient mobile robot exploration} &
                {2006} \\
            \hline
            \\
            {2.}
                {journalArticle} &
                {Xie, Li; Henkel, Christian; Stol, Karl; Xu, Weiliang} &
                {Power-minimization and energy-reduction autonomous navigation of an omnidirectional Mecanum robot via the dynamic window approach local trajectory planning} &
                {2018} \\
            \hline
            \\
            {3.}
                {conferencePaper} &
                {Wigström, Oskar; Lennartson, Bengt} &
                {Sustainable production automation-energy optimization of robot cells} &
                {2013} \\
            \hline
            \\
            {4.}
                {conferencePaper} &
                {Cheng, Haoxuan; Sato, Shimpei; Nakahara, Hiroki} &
                {A Performance Per Power Efficient Object Detector on an FPGA for Robot Operating System (ROS)} &
                {2018} \\
            \hline
            \\
            {5.}
                {conferencePaper} &
                {Licea, Daniel Bonilla; McLernon, Des; Ghogho, Mounir; Zaidi, Syed Ali Raza} &
                {An energy saving robot mobility diversity algorithm for wireless communications} &
                {2013} \\
            \hline
            \\
            {6.}
                {conferencePaper} &
                {Kırtay, Murat; Oztop, Erhan} &
                {Emergent emotion via neural computational energy conservation on a humanoid robot} &
                {2013} \\
            \hline
            \\
            {7.}
                {journalArticle} &
                {Rahman, Akhlaqur; Jin, Jiong; Rahman, Ashfaqur; Cricenti, Antonio; Afrin, Mahbuba; Dong, Yu-ning} &
                {Energy-efficient optimal task offloading in cloud networked multi-robot systems} &
                {2019} \\ 
            \hline
            \\ 
            {8.}
                {journalArticle} &
                {Gürel, Sinan; Gultekin, Hakan; Akhlaghi, Vahid Eghbal} &
                {Energy conscious scheduling of a material handling robot in a manufacturing cell} &
                {2019} \\
            \hline
            \\
            {9.}
                {conferencePaper} &
                {Huh, Sungju; Hong, Seongsoo; Lee, Joonghyun} &
                {Energy-efficient distributed programming model for swarm robot} &
                {2013} \\    
            \hline
            \\
            {10.}
                {conferencePaper} &
                {Kim, Jeongwan; Dietz, J. Eric; Matson, Eric T.} &
                {Modeling of a multi-robot energy saving system to increase operating time of a firefighting robot} &
                {2016} \\
            \hline
            \\
            {11.}
                {journalArticle} &
                {Benkrid, Abdenour; Benallegue, Abdelaziz; Achour, Noura} &
                {Multi-robot Coordination for Energy-Efficient Exploration} &
                {2019} \\
            \hline
            \\
            {12.}
                {journalArticle} &
                {Kaitwanidvilai, Somyot; Chanarungruengkij, Veerasak; Konghuayrob, Poom} &
                {Remote Sensing to Minimize Energy Consumption of Six-axis Robot Arm Using Particle Swarm Optimization and Artificial Neural Network to Control Changes in Real Time} &
                {2020} \\
            \hline
            \\
            {13.}
                {conferencePaper} &
                {Barili, A.; Ceresa, M.; Parisi, C.} &
                {Energy-saving motion control for an autonomous mobile robot} &
                {1995} \\
            \hline
            \\
            {14.}
                {conferencePaper} &
                {Jia, Menglei; Zhou, GuangMing; Chen, ZongHai} &
                {An efficient strategy integrating grid and topological information for robot exploration} &
                {2004} \\
            \hline
            \\
            {15.}
                {journalArticle} &
                {Hou, Gang; Zhou, Kuanjiu; Qiu, Tie; Cao, Xun; Li, Mingchu; Wang, Jie} &
                {A novel green software evaluation model for cloud robotics} &
                {2017} \\
            \hline
            \\
            {16.}
                {conferencePaper} &
                {Mei, Yongguo; Lu, Yung-Hsiang; Hu, Y. Charlie; Lee, CS George} &
                {A case study of mobile robot's energy consumption and conservation techniques} &
                {2005} \\      
            \hline
            \\
            {17.}
                {conferencePaper} &
                {Patel, Sonali; Shukla, Anupam; Tiwari, Ritu} &
                {Efficient strategy for co-ordinated multirobot exploration} &
                {2012} \\
        \bottomrule
    \end{tabular}
    \label{table:primary_studies}
\end{table} \clearpage
\section{Data Sheet Part 1}
\label{appendix:data_sheet_1}

\begin{table}[h]
    \centering
    \caption{Data Sheet Part 1}
    \begin{tabular}{p{0.1cm}p{3cm}p{4cm}p{4cm}p{4cm}}
        \toprule
            {ID} &
                {Energy Metric}      & 
                {QA Trade-off}       & 
                {Application Domain} &
                {Identified Major Consumers}    \\
        \midrule
            {1.} &
                {Units of Energy} &
                {Performance Efficiency vs Energy Efficiency} &
                {Robot Exploration} &
                {Stops, turns, accelerating, decellerating, sensor distance waste, 
                inefficient algorithm for path finding} \\
            \hline
            \\
            {2.} &
                {Power: Watts (W)
                Energy: Joules (J)} &
                {Mobility vs Energy Efficiency} &
                {Robot Exploration} &
                {Big change in direction} \\
            \hline
            \\
            {3.} &
                {Joules (J)} &
                {Maintainability vs Energy Efficiency} &
                {Industrial Robot} &
                {Idle time} \\
            \hline
            \\
            {4.} &
                {FPS / W (Watts)} &
                {Maintainability vs Energy Efficiency} &
                {Service Robot} &
                {Inefficient use of robot hardware} \\
            \hline
            \\
            {5.} &
                {J / M (Joules / Meter)} &
                {Performance Efficiency vs Energy Efficiency} &
                {Wireless Robot Communication} &
                {Inefficiency of sending data over bad connection} \\
            \hline
            \\
            {6.} &
                {\textbf{NOT GIVEN}} &
                {Reliability vs Energy Efficiency} &
                {Service Robot} &
                {Processing of stimuli which costs too much energy} \\
            \hline
            \\
            {7.} &
                {Joules (J)} &
                {Performance Efficiency vs Energy Efficiency} &
                {Robot Exploration} &
                {Inefficient on-board computations instead of off-loading} \\
            \hline
            \\
            {8.} &
                {KiloJoules (KJ)} &
                {Performance Efficiency vs Energy Efficiency} &
                {Industrial Robot} &
                {Idle time} \\
            \hline
            \\
            {9.} &
                {\textbf{NOT GIVEN}} &
                {Maintainability vs Energy Efficiency} &
                {Wireless Robot Communication} &
                {Redudant data transmission} \\
            \hline
            \\
            {10.} &
                {Power: W / h (Watts)
                Energy: J / h (Joules)} &
                {Maintainability vs Energy Efficiency} &
                {Firefighting Robot} &
                {Loss of friction due to weight of payload} \\
            \hline
            \\
            {11.} &
                {\textbf{NOT GIVEN}} &
                {Maintainability vs Energy Efficiency} &
                {Robot Exploration} &
                {Redundancy / Inefficiency in overlap in paths} \\
            \hline
            \\
            {12.} &
                {Joules (J)} &
                {Performance Efficiency vs Energy Efficiency} &
                {Industrial Robot} &
                {Accelerating, Maintaining Speed, Decelerating, Idle time} \\
            \hline
            \\
            {13.} &
                {Power: W / h (Watts)
                Energy: KiloJoules (KJ)} &
                {Performance Efficiency vs Energy Efficiency} &
                {Robot Exploration} &
                {Obstacle avoidance without energy in mind} \\
            \hline
            \\
            {14.} &
                {\textbf{NOT GIVEN}} &
                {Performance Efficiency vs Energy Efficiency} &
                {Robot Exploration} &
                {Inefficient path planner} \\
            \hline
            \\
            {15.} &
                {NanoJoule (nJ)} &
                {\textbf{NOT GIVEN}} &
                {Energy Consumption Analysis} &
                {Poor quality software} \\
            \hline
            \\
            {16.} &
                {Power: Watts (W)} &
                {\textbf{NOT GIVEN}} &
                {Energy Consumption Analysis} &
                {Motion is the major consumer} \\
            \hline
            \\
            {17.} &
                {Units of Energy} &
                {Performance Efficiency vs Energy Efficiency} &
                {Robot Exploration} &
                {More than one robot moving to a target, 
                Robot assigned to target is not most efficient,
                Robots colliding on their way to targets} \\
        \bottomrule
    \end{tabular}
    \label{table:data_sheet_part_1}
\end{table} \clearpage
\section{Data Sheet Part 2}
\label{appendix:data_sheet_2}

\begin{table}[h]
    \centering
    \caption{Data Sheet Part 2}
    \begin{tabular}{p{0.1cm}p{5cm}p{5cm}p{5cm}}
        \toprule
            {ID} &
                {Major Consumer Group} &
                {Identified Improving Software Aspect} &
                {Improving Aspect Group} \\
        \midrule
            {1.} &
                {Movement Related} &
                {Improved algorithm, not wasting sensor distance, limit nr of stops etc, 
                target selection based on orientation instead of utility} &
                {Improved Procedure} \\
            \hline
            \\
            {2.} &
                {Movement Related, Inefficient HW, Inefficient Procedure} &
                {Extended Dynamic Window Approach (DWA)} & 
                {Improved Procedure} \\
            \hline
            \\
            {3.} &
                {Inefficient Procedure} &
                {Scheduling algorithm reducing idle time} &
                {Improved Procedure} \\
            \hline
            \\
            {4.} &
                {Inefficient HW} &
                {Use of an FPGA} &
                {Improved Procedure, Improved HW} \\
            \hline
            \\
            {5.} &
                {Inefficient Procedure} &
                {Algorithm that would look for a better connection} &
                {Improved Procedure} \\
            \hline
            \\
            {6.} &
                {Inefficient Procedure} &
                {Trained neural network that would reject such stimuli} &
                {Improved Procedure} \\
            \hline
            \\
            {7.} &
                {Inefficient HW, Inefficient Procedure}
                {Off-loading to cloud or more available robot} &
                {Improved Procedure, Improved HW} \\
            \hline
            \\
            {8.} &
                {Inefficient Procedure} &
                {Scheduling algorithm reducing idle time} &
                {Improved Procedure} \\
            \hline
            \\
            {9.} &
                {Inefficient Procedure} &
                {Reducing redundant data} &
                {Improved Procedure} \\
            \hline
            \\
            {10.} &
                {Inefficient HW, Inefficient Procedure} &
                {Adding subrobots carrying the weight} &
                {Improved Procedure, Improved HW} \\
            \hline
            \\
            {11.} &
                {Inefficient Procedure} &
                {Improved path finder} &
                {Improved Procedure} \\
            \hline
            \\
            {12.} &
                {Movement Related} &
                {Balancing cycle time with major consumers} &
                {Improved Procedure} \\
            \hline
            \\
            {13.} &
                {Inefficient Procedure} &
                {Trajectory planning with energy in mind} &
                {Improved Procedure} \\
            \hline
            \\
            {14.} &
                {Inefficient Procedure} &
                {Improved path planner} &
                {Improved Procedure} \\
            \hline
            \\
            {15.} &
                {Inefficient Procedure} &
                {Ability to analyse energy consumption of software} &
                {Improved Procedure} \\
            \hline
            \\
            {16.}  &
                {Movement Related} &
                {\textbf{NOT GIVEN - Analysis}} &
                {\textbf{NOT GIVEN - Analysis}} \\
            \hline
            \\
            {17.}  &
                {Inefficient Procedure} &
                {An algorithm preventing given consumers} &
                {Improved Procedure} \\
        \bottomrule
    \end{tabular}
    \label{table:data_sheet_part_2}
\end{table} \clearpage
\section{Data Sheet Part 3}
\label{appendix:data_sheet_3}

\begin{table}[h]
    \centering
    \caption{Data Sheet Part 3}
    \begin{tabular}{p{0.1cm}p{5cm}p{2cm}p{2cm}p{3.5cm}p{3cm}}
        \toprule
            {ID} &
                {Major Contribution} &
                {Experiment}         &
                {Comparison}         &
                {Energy Model}       &
                {Energy Model Group} \\
        \midrule
            {1.} &
                {The improved algorithm} &
                {Simulation} &
                {Yes} &
                {1 Unit of Distance = 1 Unit of Energy.
                Each stop = 0.5 Units of Energy.
                Each 45° degree turn = 0.4 Units of Energy,
                Each additional 45° of turn adds 0.2 units of Energy.} &
                {Abstract, Simple} \\
            \hline
            \\
            {2.} &
                {Extended DWA with Energy Cost} &
                {Simulation} &
                {Yes} &
                {Mathematical Kinematics} &
                {Representational, Complex} \\
            \hline
            \\
            {3.} &
                {The scheduling algorithm} &
                {Simulation} &
                {Yes} &
                {Page 2 to 4} &
                {Representational, Complex} \\
            \hline
            \\
            {4.} &
                {The complete system; using an FPGA in combination with a neural network} &
                {Real-world} &
                {Yes} &
                {Page 2} &
                {Representational, Complex} \\
            \hline
            \\
            {5.} &
                {Elaborate algorithm, finding best transmit location} &
                {Simulation} &
                {Yes} &
                {Page 2 to 4} &
                {Representational, Complex} \\
            \hline
            \\
            {6.} &
                {The actual trained neural network} &
                {Simulation} &
                {No} &
                {Page 3} &
                {Representational, Complex} \\
            \hline
            \\
            {7.} &
                {An algorithm off-loading to more available robots or the cloud when more efficient} &
                {Simulation} &
                {Yes} &
                {Page 8 to 11 AND page 13} &
                {Representational, Complex} \\
            \hline
            \\
            {8.} &
                {The scheduling algorithm} &
                {Real-world} &
                {Yes} &
                {Page 3 to 7} &
                {Representational, Complex} \\
            \hline
            \\
            {9.} &
                {Applying distributed systems paradigm MapReduce to reduce redundant data} &
                {Simulation} &
                {Yes} &
                {\textbf{NOT GIVEN}} &
                {None} \\
            \hline
            \\
            {10.} &
                {The complete system of subrobots, with the distributed systems algorithm to 
                facilitate communication} &
                {Simulation} &
                {Yes} &
                {\textbf{NOT GIVEN}} &
                {None} \\
            \hline
            \\
            {11.} &
                {The improved path finder} &
                {Combination} &
                {Yes} &
                {page 3 to 4} & 
                {Representational, Complex} \\
            \hline
            \\
            {12.} &
                {Algorithm balancing cycle time with major consumers} &
                {Simulation} &
                {Yes} &
                {page 3 to 4} &
                {Representational, Complex} \\
            \hline
            \\
            {13.} &
                {On-board trajectory planning with energy in mind} &
                {None} &
                {No} &
                {\textbf{NOT GIVEN}} &
                {None} \\
            \hline
            \\
            {14.} &
                {Path planner with CostOverflow (CO) added, to get optimal time-energy paths} &
                {Combination} &
                {Yes} &
                {Page 3 to 4} &
                {Representational, Complex} \\
            \hline
            \\
            {15.} &
                {A novel green evaluation method for software} &
                {Combination} &
                {Yes} &
                {\textbf{NOT GIVEN}} &
                {None} \\
            \hline
            \\
            {16.}  &
                {A detailed study, identifying major consumers} &
                {Real-world} &
                {No} &
                {Page 2 to 4} & 
                {Representational, Complex} \\
            \hline
            \\
            {17.}  &
                {The entire system, including the algorithm mentioned} &
                {Simulation} &
                {Yes} &
                {Same as study 1} & 
                {Abstract, Simple} \\
        \bottomrule
    \end{tabular}
    \label{table:data_sheet_part_3}
\end{table}

\end{document}
